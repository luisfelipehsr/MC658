\documentclass[12pt]{article}
% multirow allows you to combine rows in columns
\usepackage{multirow}
% tabularx allows manual tweaking of column width
\usepackage{tabularx}
% longtable does better format for tables that span pages
\usepackage{longtable}
\usepackage[utf8]{inputenc}
\usepackage{amssymb}
\usepackage{latexsym}
\usepackage{mathtools}
\usepackage{amsmath}
\renewcommand{\O}[1]{$\mathcal{O}(#1)$}
\usepackage[T1]{fontenc} 
\usepackage [utf8] {inputenc}
\title{\textbf{Relatório MC658\\\large Problema do Planejamento de Cronograma com Custo de Espera Mínimo}}
\author{Luís Felipe Hamada Serrano 147091\\Raphael Marques Franco 035501\\\\ Universidade de Campinas}

\begin{document}

\maketitle 
\today{}
% * <poli.unicamp@gmail.com> 2017-11-21T04:55:57.586Z:
%
% ^.
% * <poli.unicamp@gmail.com> 2017-11-21T04:55:53.848Z:
%
% ^.

\newpage
\tableofcontents
\newpage

\section{Introdução}

blablablabla

\newpage
\section{Heurística}

\newpage
\section{Programação Linear Inteira}
\subsection{Modelo}
\[
	\begin{array}{r@{}r@{}r@{}l}
    \textbf{Função objetivo:}&\qquad 
	H =\text{min}\; \displaystyle\sum_{i\in Atores} {(l_i - e_i+1- s_i)*c_i}\\\\
    \textbf{sujeitos às restrições:}\\ 
   \textit{Cenas: (cena tem um único dia escolhido)} &\forall j \in Cenas: \;\displaystyle\sum_{k \in Cenas} {g_{j,k} = 1} & \\\\
   \textit{Dias: (dia tem uma única cena escolhida) }&\forall k \in Cenas: \;\displaystyle\sum_{j \in Cenas} {g_{j,k} = 1} & \\\\
\textit{Gravação: (dia da gravação da cena j)}&\forall j \in Cenas:\; d_j=\displaystyle\sum_{k \in Cenas} {g_{j,k}*k} & \\\\
\textit{Gravação inicial: }\\&\forall i \in Atores\text{ e }\forall  j\in Cenas \text{ e } T_{i,j} = 1:\; e_i \leq d_j & \\\\ 
\textit{Gravação final: }\\&\forall i \in Atores\text{ e }\forall  j\in Cenas \text{ e } T_{i,j} = 1:\; d_j \leq l_i & \\\\      
Cenas = \{1, \dots ,n\} ,\:
Atores = \{1, \dots ,m\} ,c_i \in C\:
&\\
g_{j,k}\;,\; T_{i,j} \in \{0,1\},\:e_i \;,\; l_i\;,\; d_i\;,\; s_i \in Cenas \:
&\\
    	\end{array}
\] 
\subsection{Nomenclatura}
\textbf{Variavéis de Decisão}
\begin{itemize}
  \item $l_i$ : último dia escolhido para atuação do ator i;
  \item $e_i$ : primeiro dia escolhido para atuação do ator i;
  \item $g_{j,k}$ : variável binária que indica 1, se a cena j sucede imediamente após a cena k ou 0, caso contrário;
  \item $d_j$ : dia de gravação da cena j.
\end{itemize}
\newpage
\textbf{Parâmetros}
\begin{itemize}
  \item $C$ : vetor de custos pagos para cada um dos m atores na locação;
  \item $c_i$ : elemento do vetor $C$ que indica custo pago por dia de atuação ou espera em locação ao ator i;
  \item $s_i$ : total de dias de atuação do ator i na locação;
  \item $T_{i,j}$ : é um elemento da matriz T e indica se o ator i participa da cena j, 1, em caso afirmativo e 0, caso contrário.
  
  
\end{itemize}

\newpage
\section{Branch and Bound}


\vfill

\begin{thebibliography}{1}
\bibitem{s} Efternamn, Namn, (År). Titel. Företag/Förening/Anstalt/Tidsskrift
\end{thebibliography} 
\newpage
\appendix
\end{document}

