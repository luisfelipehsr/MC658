%Preâmbulo
\documentclass[12pt,a4paper]{article}
%PACOTE DO IMECC
\usepackage{centernot}
\usepackage{logo-unicamp}
\usepackage{alltt}
\usepackage{wrapfig}
\usepackage{listings}
\usepackage{subfloat}
\usepackage{url}
\usepackage{cite}
\usepackage{multirow}
\usepackage[utf8]{inputenc}
\usepackage{textcomp}
\usepackage[portuguese]{babel}
\usepackage[T1]{fontenc}
\usepackage{amsmath}
\usepackage{amssymb}
\usepackage[lofdepth]{subfig}
\usepackage{listingsutf8}%linguagem de programação
\usepackage[table,xcdraw]{xcolor}
\usepackage{color}
\usepackage{graphicx}
\usepackage[export]{adjustbox}
\usepackage{wrapfig}
\graphicspath{ {imagens/} } 
\definecolor{mygreen}{rgb}{0,0.6,0}
\definecolor{mygray}{rgb}{0.5,0.5,0.5}
\definecolor{mymauve}{rgb}{0.58,0,0.82}
\usepackage[right=2.5cm,left=2.5cm,top=2.5cm,bottom=2.5cm]{geometry}
\usepackage{indentfirst}
\usepackage{float}
\usepackage{enumerate}
\usepackage{marginnote}
%REDEFINE O COMANDO RESUMO
\renewcommand\thesubsubsection{\Roman{subsubsection}} 
\newcommand{\resumo}[2]{\noindent{\textsl{\space#1: }#2}\\}
\usepackage[hidelinks]{hyperref}
%Preparando para o PSEUDOCODIGO
\usepackage{caption}
\usepackage{algorithm}
\usepackage{algpseudocode}
\makeatletter
\algnewcommand{\LineComment}[1]{\Statex \hskip\ALG@thistlm \(\triangleright\) #1}
\makeatother
\makeatletter
\renewcommand{\ALG@name}{Algoritmo}
\makeatother
\makeatletter
\renewcommand{\listalgorithmname}{Lista de Algoritmos}
\makeatother    
% Declaracoes em Português
\floatname{algorithm}{Algoritmo}
\DeclareCaptionLabelFormat{something}{#2.#1.}
\makeatletter
 \renewcommand{\algorithmicrequire}{\textbf{\color{blue}{Entrada: }}}
 \renewcommand{\algorithmicensure}{\textbf{\color{violet}{Saída: }}}
 \algrenewcommand\algorithmicend{\textbf{fim}}
 \algrenewcommand\algorithmicdo{\textbf{faça}}
 \algrenewcommand\algorithmicwhile{\textbf{enquanto}}
 \algrenewcommand\algorithmicfor{\textbf{para}}
 \algrenewcommand\algorithmicif{\textbf{se}}
 \algrenewcommand\algorithmicthen{\textbf{então}}
 \algrenewcommand\algorithmicelse{\textbf{senão}}
 \algrenewcommand\algorithmicreturn{\textbf{devolve}}
 \algrenewcommand\algorithmicfunction{\textbf{função}}
 \algrenewcommand\algorithmicprocedure{\textbf{procedimento}}
% % Rearranja os finais de cada estrutura
 \algrenewtext{EndWhile}{\algorithmicend\ \algorithmicwhile}
 \algrenewtext{EndFor}{\algorithmicend\ \algorithmicfor}
 \algrenewtext{EndIf}{\algorithmicend\ \algorithmicif}
 \algrenewtext{EndFunction}{\algorithmicend\ \algorithmicfunction}
% O comando For, a seguir, retorna 'para #1 -- #2 até #3 faça'
\algnewcommand\algorithmicto{\textbf{até}}
\algrenewtext{For}[3]%
{\algorithmicfor\ #1 $\gets$ #2 \algorithmicto\ #3 \algorithmicdo}
\makeatother 
%Pacote para colocaração de código no uso do MINTED
\usepackage{footnote}
\usepackage{etoolbox}
\usepackage{tcolorbox}
\tcbuselibrary{minted,skins}
\newtcblisting{myc}{
  listing engine=minted,
  colback=bg,
  colframe=black!70,
  listing only,
  minted style=colorful,
  minted language=c,
  minted options={linenos=true,texcl=true},
  left=1mm,
}
\renewcommand\listingscaption{Implementação}
\renewcommand\listoflistingscaption{Lista de Implementações}
\author{Raphael Marques Franco}